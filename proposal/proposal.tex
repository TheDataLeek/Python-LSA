\documentclass[10pt]{article}

%%%%%%%%%%%%%%%%%%%%%%%%%%%%%%%%%%%%%%%%%%%%%%%%%%%%%%%%%%%%%%%%%%%%%%%%%%%%%%%%
% LaTeX Imports
%%%%%%%%%%%%%%%%%%%%%%%%%%%%%%%%%%%%%%%%%%%%%%%%%%%%%%%%%%%%%%%%%%%%%%%%%%%%%%%%
\usepackage{amsfonts}                                                   % Math fonts
\usepackage{amsmath}                                                    % Math formatting
\usepackage{amssymb}                                                    % Math formatting
\usepackage{amsthm}                                                     % Math Theorems
%\usepackage{arydshln}                                                   % Dashed hlines
\usepackage{attachfile}                                                 % AttachFiles
\usepackage{cancel}                                                     % Cancelled math
\usepackage{caption}                                                    % Figure captioning
\usepackage{color}                                                      % Nice Colors
\usepackage[at]{easylist}                                        % Easy lists
\usepackage{fancyhdr}                                                   % Fancy Header
\usepackage[T1]{fontenc}                                                % Specific font-encoding
\usepackage[margin=1.5in, marginparwidth=2cm, marginparsep=2cm]{geometry} % Margins
\usepackage{graphicx}                                                   % Include images
\usepackage{hyperref}                                                   % Referencing
\usepackage[none]{hyphenat}                                             % Don't allow hyphenation
\usepackage{lipsum}                                                     % Lorem Ipsum Dummy Text
\usepackage{listings}                                                   % Code display
\usepackage{marginnote}                                                 % Notes in the margin
\usepackage{microtype}                                                  % Niceness
\usepackage{multirow}                                                   % Multirow tables
\usepackage[framemethod=tikz]{mdframed}                                 % background color
\usepackage{pdfpages}                                                   % Include pdfs
\usepackage{pgfplots}                                                   % Create Pictures
\usepackage{rotating}                                                   % Figure rotation
\usepackage{setspace}                                                   % Allow double spacing
%\usepackage{subcaption}                                                 % Figure captioning
\usepackage{subfig}                                                 % Figure captioning
%\usepackage{tocloft}                                                    % List of Equations
\usepackage{longtable}                                                  % Huge Tables
\usepackage{supertabular}
\usepackage{float}
\setcounter{LTchunksize}{50}
%%%%%%%%%%%%%%%%%%%%%%%%%%%%%%%%%%%%%%%%%%%%%%%%%%%%%%%%%%%%%%%%%%%%%%%%%%%%%%%%
% Package Setup
%%%%%%%%%%%%%%%%%%%%%%%%%%%%%%%%%%%%%%%%%%%%%%%%%%%%%%%%%%%%%%%%%%%%%%%%%%%%%%%%
\hypersetup{%                                                           % Setup linking
    colorlinks=true,
    linkcolor=black,
    citecolor=black,
    filecolor=black,
    urlcolor=black,
}
\RequirePackage[l2tabu, orthodox]{nag}                                  % Nag about bad syntax
\renewcommand*\thesection{\arabic{section}}                             % Reset numbering
\renewcommand{\footrulewidth}{0.4pt}                                    % Footer hline
\setcounter{secnumdepth}{3}                                             % Include subsubsections in numbering
\setcounter{tocdepth}{3}                                                % Include subsubsections in toc
%%%%%%%%%%%%%%%%%%%%%%%%%%%%%%%%%%%%%%%%%%%%%%%%%%%%%%%%%%%%%%%%%%%%%%%%%%%%%%%%
% Custom commands
%%%%%%%%%%%%%%%%%%%%%%%%%%%%%%%%%%%%%%%%%%%%%%%%%%%%%%%%%%%%%%%%%%%%%%%%%%%%%%%%
\newcommand{\nvec}[1]{\left\langle #1 \right\rangle}                    %  Easy to use vector
\newcommand{\inprod}[2]{\left\langle \vec{#1}, \vec{#2} \right\rangle}  %  Easy to use inner product
\newcommand{\norm}[1]{\lvert \lvert \vec{#1} \rvert \rvert}             %  Easy to use norm
\newcommand{\ma}[0]{\mathbf{A}}                                         %  Easy to use vector
\newcommand{\mb}[0]{\mathbf{B}}                                         %  Easy to use vector
\newcommand{\abs}[1]{\left\lvert #1 \right\rvert}                       %  Easy to use abs
\newcommand{\pren}[1]{\left( #1 \right)}                                %  Big parens
\newcommand{\Var}[0]{\text{Var}}                                %  Variance
\newcommand{\Cov}[0]{\text{Cov}}                                %  Variance
\newcommand{\Corr}[0]{\text{Corr}}                                %  Variance
\let\oldvec\vec
\renewcommand{\vec}[1]{\mathbf{#1}}                            %  Vector Styling
\newtheorem{thm}{Theorem}                                               %  Define the theorem name
\theoremstyle{definition}
\newtheorem{definition}{Definition}                                     %  Define the definition name
\newtheorem{ex}{Example}                                                %  Define the example name
\definecolor{bg}{rgb}{0.95,0.95,0.95}

\usepackage{fancyvrb}
%%%%%%%%%%%%%%%%%%%%%%%%%%%%%%%%%%%%%%%%%%%%%%%%%%%%%%%%%%%%%%%%%%%%%%%%%%%%%%%%
% Beginning of document items - headers, title, toc, etc...
%%%%%%%%%%%%%%%%%%%%%%%%%%%%%%%%%%%%%%%%%%%%%%%%%%%%%%%%%%%%%%%%%%%%%%%%%%%%%%%%
\pagestyle{fancy}                                                       %  Establishes that the headers will be defined
\fancyhead[LE,LO]{APPM4720/5720 Project Proposal}                                  %  Adds header to left
\fancyhead[RE,RO]{Rishabh Raghavendran - William Farmer}                                       %  Adds header to right
\cfoot{\thepage}
\lfoot{APPM4720/5720}
\rfoot{Gunnar Martinsson}
\title{APPM4720/5720 Project Proposal}
\author{Rishabh Raghavendran - William Farmer}
%%%%%%%%%%%%%%%%%%%%%%%%%%%%%%%%%%%%%%%%%%%%%%%%%%%%%%%%%%%%%%%%%%%%%%%%%%%%%%%%
% Beginning of document items - headers, title, toc, etc...
%%%%%%%%%%%%%%%%%%%%%%%%%%%%%%%%%%%%%%%%%%%%%%%%%%%%%%%%%%%%%%%%%%%%%%%%%%%%%%%%
\begin{document}

\maketitle

\begin{easylist}[enumerate]
    @ Introduction:
    @@ Overview:

    Finding the meaning of documents and drawing relations between different
    documents is an extremely human activity. We as humans possess the innate
    ability to reason and make associations based on this. In today's age of
    automation and artificial intelligence, is it possible to replicate this
    behaviour using a computer to any extent? We believe it is and the aim of our
    project is to use some smart mathematics to do exactly that. But before we move
    forth we need to actually decide whether practicality to doing this. Why would
    we even want the computer to replicate this sort of human behavior?  Aside from
    scholastic and academic curiosity about the challenge in solving this problem,
    there are bona fide reasons for doing this; the primary one being that computers
    are much better than humans at doing repetitive tasks many times.  While it
    would take a human many days to go through a large set of documents, a computer
    would theoretically be able to do that in a fraction of the time. 

    @@ Our proposed Solution:

    To solve the aforementioned problem, we want to perform Latent Semantic
    Analysis using the Singular Value Decomposition of the documents and the
    associated document-term matrices. Our preliminary research has shown that
    by using this algorithm, we can not only query for certain keywords in
    documents, but also  find the documents that are in some way related  to the
    query word but do not explicitly contain it.\footnote{%
    For example, suppose we have the following set of five documents and a
    search query: dies, dagger.

    \begin{table}[H]
        \centering
        \begin{tabular}{ll}
            d1 & Romeo and Juliet. \\
            d2 & Juliet: O happy dagger! \\
            d3 & Romeo died by dagger. \\
            d4 & ``Live free or die'', that's the New-Hampshire's motto. \\
            d5 & Did you know, New-Hampshire is in New-England. \\
        \end{tabular}
    \end{table}
 
    Clearly, d3 should be ranked top of the list since it contains both dies,
    dagger. Then, d2 and d4 should follow, each containing a word of the query.
    However, what about d1 and d5? Should they be returned as possibly
    interesting results to this query? As humans we know that d1 is quite
    related to the query. On the other hand, d5 is not so much related to the
    query. Thus, we would like d1 but not d5, or differently said, we want d1 to
    be ranked higher than d5.}

    The question is, can a Machine do this? The answer is Yes, LSA does exactly
    that.

    @@ Our Dataset and Metric for success

    So far we have come up with an interesting problem to solve and an ingenious
    technique to solve it, but are there any real world datasets for which this
    problem is a viable one? In fact, the internet is full of these kind of
    datasets and for this project we will explore a few of the ones we find most
    interesting. For example, we can obtain a dataset of all the recipes in a
    book and then query for specific ingredients or cuisines. For example,
    suppose we wish to make some thai food with chicken, rice and a few other
    ingredients, we can search the entire database of recipes and arrange them
    according to what matches perfectly with our query, and others which are not
    exactly that but fairly similar. 

    Another interesting dataset for which this problem would yield some interesting
    results is the database of the inaugural addresses of each of the presidents of
    the USA to find what are the words and related topics most spoken about by them
    and draw comparisons between their speeches and the speeches of other famous
    people.


    @ Latent Semantic Analysis using SVDs:
    @@ What is it?

    LSA (or sometimes referred to as Latent Semantic Indexing (LSI)), is a
    method to discover hidden concepts in document data. It does this by
    analyzing relationships between a set of documents and the terms they
    contain by producing a set of concepts related to the documents and terms.

    LSA presumes that words that are close in  meaning will occur in similar
    pieces of text. 

    @@ Implementation

    First, we will construct a document-term matrix from a large piece of text. We will then perform dimensional reduction using singular value decomposition so as to reduce the number of rows while preserving the structural similarity among columns. Words are represented as normalised vectors and are compared by taking the dot product between them. 
    We wish to implement this in Python. The reasons for that are: 

    @@@ Ease of implementation
    @@@ A vast library of resources
    @@@ Author Familiarity

    @ Additional Techniques and Modifications:

    Besides the aforementioned techniques using LSA, we also will explore other
    techniques for large-scale document analysis and natural language
    processing. Note, some of these extensions and modifications are intertwined
    and rely on many of the same core concepts. In this section we will also
    discuss the qualities of the different methods versus our initial LSA
    implementation.

    @@ Cluster analysis and Non-negative matrix factorization

    Using matrix factorization of the document-term matrix we can factor and
    establish clusters. Analysis can then be performed on these clusters for
    similarity and term searching.

    @@ Probabilistic Latent Semantic Analysis

    An extended version of LSA, PLSA uses multinomial distributions to model the
    behavior of our documents using a multinomial distribution.

    @@ Other extensions

    As we perform our research, we will undoubtedly come across several other
    well-studied methods to perform this type of natural language processing,
    and they may be added at a later time.
\end{easylist}

\end{document}
